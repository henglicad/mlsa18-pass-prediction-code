\section{Introduction} \label{intro}
%This section discuss our motivation, briefly mention our approach, and introduce the research questions that we want to answer:
In a football game, players pass the ball to their teammates in order to create good shooting opportunities or prevent the opposing team from getting the control of the ball.
Accurately passing ball to the right player is essential for winning a football game~\cite{Ali2011Measuring,Hughes2005Analysis}.

Prior work~\cite{reep1968skill,Hughes2005Analysis} studies how passing sequences lead to goals. Their findings have shaped the tactics of many football coaches.
In this work, we study football players' passing patterns and construct machine learning models to anticipate passing targets.
We believe that football coaches and players can take our results into consideration when they make their tactics or make their passes/runs. 
%For example, anticipating passing targets can help players be aware of the best positions to get a pass.
Anticipating targets of passes can also help automatic cameras to always focus on the ball in a game.

This work analyzes a dataset which contains 12,124 passes performed by a Belgian football club in 14 games. We want to answer the following research questions: 

\begin{description}
	\item \textbf{RQ1: What are players' passing patterns?}
	We discuss how players pass the ball on the field, e.g., what is their passing accuracy, how often do they pass the ball forwards?
	Understanding players' passing patterns can help us derive features that can explain players' passing targets. 
	%Players' passing patterns can also help football coaches and players make informed tactics.
	
	\item \textbf{RQ2: How well can we model passing targets?}
	We construct machine learning models to predict the target of a football pass. 
	An accurate model can help coaches and players make informed tactics.
	
	\item \textbf{RQ3: What are the influential factors that explain the passing targets?}
	We analyze the models to find the the most influential factors that explain the target of a pass. Understanding such influential factors can help coaches and players improve their tactics according to these factors.
\end{description}

\textbf{Paper organization.}
The remainder of the paper is organized as follows.
Section~\ref{methodology} discusses our approaches for constructing our prediction models, including feature extraction, feature selection, modeling approaches and our evaluation measures.
Section~\ref{results} present the results for answering our research questions.
Finally, Section~\ref{conclusions} draws conclusions.

