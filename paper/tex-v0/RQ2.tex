\subsection{RQ2: how well can we model passing targets?}\label{RQ2-results}

\textbf{Our models can predict the target of a pass with a top-1 accuracy of 49\% and a top-5 accuracy of 94\%, when we exclude false passes (i.e., passes to the other team).}
Table~\ref{tab:performance-accurate-passes} shows the performance of our models in terms of \textit{top-n} accuracy when we exclude false passes. 
The top-1, top-3, and top-5 accuracy is 49\%, 83\%, and 94\%, respectively, which means the actual target of a pass has a 94\% chance of being within our top-5 predicted candidates.

\begin{table}[!t]
\caption{The accuracy of our models for predicting passing targets (excluding false passes).}
\centering
\begin{tabular}{lcccc}
  \hline
  & Back-field & Middle-field & Front-field & Overall \\
  \hline
  Top-1 accuracy & 51\% & 45\% & 54\% & 49\% \\
  Top-3 accuracy & 83\% & 81\% & 90\% & 83\% \\
  Top-5 accuracy & 93\% & 93\% & 97\% & 94\% \\
  MRR & 0.69 & 0.64 & 0.73 & - \\
  \hline
\end{tabular}
\label{tab:performance-accurate-passes}
\end{table}

\textbf{Our models can predict the target of a pass with a top-1 accuracy of 41\% and a top-5 accuracy of 81\%, when we consider all passes.}
Table~\ref{tab:performance-all-passes} shows the performance of our models when we consider all passes (including false passes). 
The performance of our models decreases when we consider false passes (i.e., passes to the other team). 
False passes are very difficult to predict because it is not the sender player's intention to pass the ball to the other team. 

\begin{table}[!t]
\caption{The accuracy of our models for predicting passing targets (considering all passes including passes to the other team).}
\centering
\begin{tabular}{lcccc}
  \hline
  & Back-field & Middle-field & Front-field & Overall \\
  \hline
  Top-1 accuracy & 44\% & 38\% & 42\% & 41\% \\
  Top-3 accuracy & 72\% & 67\% & 71\% & 70\% \\
  Top-5 accuracy & 82\% & 79\% & 82\% & 81\% \\
  MRR & 0.60 & 0.56 & 0.59 & 0.58 \\

  \hline
\end{tabular}
\label{tab:performance-all-passes}
\end{table}

\textbf{Our models perform better when the sender of a pass is in the back and front areas of the field, while perform worse when the sender is in the middle area of the field.}
Table~\ref{tab:performance-accurate-passes} and Table~\ref{tab:performance-all-passes} also shows the performance of our models for the passes when the sender is in the back, middle and front areas of the field, separately.
Surprisingly, the performance of our models is worst when the sender is in the middle area of the field. A player in the middle area may have more passing options, thereby increasing the difficulty to predict the right targets.