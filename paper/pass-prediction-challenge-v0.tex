% This is samplepaper.tex, a sample chapter demonstrating the
% LLNCS macro package for Springer Computer Science proceedings;
% Version 2.20 of 2017/10/04
%
\documentclass[runningheads]{llncs}
%
\usepackage{graphicx}
% Used for displaying a sample figure. If possible, figure files should
% be included in EPS format.
%
% If you use the hyperref package, please uncomment the following line
% to display URLs in blue roman font according to Springer's eBook style:
% \renewcommand\UrlFont{\color{blue}\rmfamily}

\usepackage{color}
\newcommand{\heng}[1]{\textcolor{blue}{{\it [Heng says: #1]}}}
\newcommand{\todo}[1]{\textcolor{red}{\textbf{[[TODO: #1]]}}}

\usepackage{fancybox}		
\newcommand{\hypobox}[1]{\begin{center}%	
	\noindent\thicklines\setlength{\fboxsep}{7pt}%	
	%\cornersize{0}\Ovalbox{\begin{minipage}{3.1in}%
	%\cornersize{0}\shadowbox{\begin{minipage}{3.1in}%	
	%\cornersize{0}\Ovalbox{\begin{minipage}{4in}%
	\cornersize{0}\Ovalbox{\begin{minipage}{0.96\textwidth}%	
		%\vspace{-0.1cm}
		%\textit{#1}
		#1
		%\vspace{-0.1cm}
		\end{minipage}} \end{center}}

%\usepackage{booktabs} % For formal tables


\begin{document}
%
\title{Predicting the Receivers of Football Passes}
%
%\titlerunning{Abbreviated paper title}
% If the paper title is too long for the running head, you can set
% an abbreviated paper title here
%
\author{Heng Li\inst{1} \and
Zhiying Zhang\inst{2}
% \and Third Author\inst{3}\orcidID{2222--3333-4444-5555}
 }
%
\authorrunning{H. Li and Z. Zhang}
% First names are abbreviated in the running head.
% If there are more than two authors, 'et al.' is used.
%
\institute{School of Computing, Queen's University, Kingston, Canada\\
\email{hengli@cs.queensu.ca}
\and
Microsoft, Bellevue, WA, USA\\
\email{zhiyingz@microsoft.com}\\
}
%
\maketitle              % typeset the header of the contribution
%
\begin{abstract}
%The abstract should briefly summarize the contents of the paper in
%150--250 words.
Football (or association football) is a highly-collaborative team sport. 
Passing the ball to the right person is essential for winning a football game.
Anticipating the target of a pass can help football players build better collaborations on the field and help coaches make better tactic decisions.
In this work, we use machine learning models to learn how professional football players pass the ball, and leverage the models to predict the targets of football passes.
Our models can predict the passing target with a top-1 accuracy of 41\% and a top-5 accuracy of 82\% (\todo{numbers to be updated}).
This paper discusses the features that we use to build our models, our modeling approaches, and the important factors that explain the the target of a pass.


\keywords{Football pass prediction \and Learning to rank \and LightGBM.}
\end{abstract}
%
%
%

\section{Introduction} \label{intro}
%This section discuss our motivation, briefly mention our approach, and introduce the research questions that we want to answer:
In a football game, players pass the ball to their teammates in order to create good shooting opportunities or prevent the opposing team from getting the control of the ball.
Accurately passing ball to the right player is essential for winning a football game~\cite{Ali2011Measuring,Hughes2005Analysis}.

Prior work~\cite{reep1968skill,Hughes2005Analysis} studies how passing sequences lead to goals. Their findings have shaped the tactics of many football coaches.
In this work, we study football players' passing patterns and construct machine learning models to anticipate passing targets.
We believe that football coaches and players can take our results into consideration when they make their tactics or make their passes/runs. For example, anticipating passing targets can help players be aware of the best positions to get a pass.

This work analyzes a dataset which contains 12,124 passes performed by a Belgian football club in 14 games. We want to answer the following research questions: 

\begin{description}
	\item \textbf{RQ1: what are players' passing patterns?}
	
	For example, most of the passes are passing forwards.
	
	\item \textbf{RQ2: how well can we model passing targets?}
	\item \textbf{RQ3: what are the influential factors that explain the passing targets?}
\end{description}

\section{Methodology} \label{methodology}
This section discuss our overall methodology, including our feature extraction process, modeling and evaluation approaches.

\subsection{Feature extraction}

From the dataset that contains information about 12,124 passes\footnotemark[\ref{origin_dataset}], we extract five dimensions of features to explain the likelihood of passing the ball to a certain receiver. In total, we extract 54 features. A full list of our features is available at our public git repository\footnote{\label{feature-list}https://github.com/henglicad/mlsa18-pass-prediction/blob/master/feature-list.md}. 
We also share our extracted feature values online~\footnote{\label{feature-values}https://github.com/henglicad/mlsa18-pass-prediction/blob/master/features.tsv}.
%Section~\ref{RQ3-results} discusses the most important features for explaining the receiver of a pass.
\begin{itemize}
	\item \textbf{Sender position features.} This dimension of features capture the position of the sender on the field, such as the sender's distance to the other team's goal. We choose this dimension of features because players have different passing strategies at different positions, for example, players may pass the ball more conservatively in their own half but more aggressively in the other team's half.
	\item \textbf{Candidate receiver position features.} This dimension of features capture the position of a candidate receiver, such as the candidate receiver's distance to the sender. Senders always consider candidate receivers' positions when they decide to whom to pass the ball.
	\item \textbf{Passing path features.} This dimension of features measure the quality of a passing path (i.e., the path from the sender to a candidate receiver), such as the passing angle. The quality of a passing path can predict the outcome (success/failure) of a pass.
	\item \textbf{Team position features.} This dimension of features capture the overall position of the team in control of the ball, such as the front line of the team. Team position might also impact the passing strategy, for example, a defensive team position might be more likely to pass the ball forwards.
	\item \textbf{Game state features.} This dimension of feature capture the state of the whole game, such as the time when the sender passes the ball. \textbf{We do not use the time when the receiver receives the ball as a feature in our model, as it exposes information about the actual pass (e.g., pass duration).}
\end{itemize}

%\subsection{Removing redundant features}

%Redundant features usually add more complexity to the model than the information they provide to the model. Redundant features can also result in highly unstable models~\cite{kuhn2013applied}.
%In this work, we calculate the pairwise Spearman correlation between our extracted features and remove collinearity among these features.
%If the correlation between a pair of features is greater than a threshold, we only keep one of the two features in our model.
%In this work, we choose the correlation value of 0.8 as the threshold to remove collinear metrics, as suggested by prior work~\cite{kuhn2013applied}.

\subsection{Modeling approach}

We formulate the task of predicting the receiver of a football pass as a learning to rank problem~\cite{liu2009learning}. 
For each pass, our learning to rank model outputs a ranked list of the candidate receivers. 
A good model should rank the correct receiver in the front of the ranked list.
Gradient boosting decision tree (GBDT) is widely used for learning to rank tasks.
There are quite a few effective implementations of GBDT, such as XGBoost and pGBRT, which usually achieves state-of-the-art performance in learning to rank tasks.

In this work, we use an efficient implementation of GBDT, \textbf{LightGBM}~\cite{NIPS2017_6907}, which speeds up the training time of conventional GBDT (e.g., XGBoost and pGBRT) by up to 20 times while achieving almost the same accuracy. 
We use an open source implementation of LightGBM that is contributed by Microsoft\footnote{https://github.com/Microsoft/LightGBM}.

%\todo{Please help refine the language}
We use a 10-fold cross-validation approach to build and test our model. 
All the passes in the dataset are randomly partitioned into 10 subsets of roughly equal size. 
%One subset is used as testing set (i.e., the held-out set) and the other nine subsets are used as training set. 
We build our model using nine subsets (i.e., the model building data) and evaluate the performance of our model on the held-out subset (i.e., the testing data).
%We train our models using the training set and evaluate the performance of our model on the held-out set.
The process repeats 10 times until all subsets are used as testing data once.

In each fold, we further split the model building data into the training data and validation data.
We train the model on the training data and use the validation data to tune the hyper-parameters of the model. % \todo{list all the hyperparameters}.
We do a grid search to get the top three sets of hyper-parameter values according to the performance of the model on the validation data.
Then, we build three models with these three set of hyper-parameters using the training data. 
We apply these three models on the testing data and get three sets of results.
We then average the results for each receiver candidate and use the averaged results to rank the receiver candidates. We find that with such a ensemble modeling approach, the accuracy of our model improves up to 2\%.% \todo{shall we use relative improvement?}.



%For each fold, we hold out 10\% of the data as test set. 
%For the first fold, we continue to split the data into training and validation data, then we do a grid search to get the top three set of parameters according to the validation set performance. Then for each fold, we train three models with these three set of hyperparameters \todo{list all the hyperparameters} on the training data. On test data, we use the three models to predict to get three sets of result, and then average the prediction results for each candidate, then use the averaged result to do ranking. We find that with model ensembling, all the accuracy numbers improve about 0.1\%\~2.0\%.

\subsection{Evaluation approaches}

We use \textbf{top-N accuracy} and \textbf{mean reciprocal rank (MRR)} to measure the performance of our model.
Top-N accuracy measures the accuracy of the model's top-N recommendations, i.e., the probability that the correct receiver of a pass appears in the top-N receiver candidates that are predicted by the model.
For example, top-1 accuracy measures the probability that the correct receiver of a pass is the first player in the predicted list of receiver candidates.

Reciprocal rank is the inverse of the rank of the correct receiver of a pass in an ranked list of candidate receivers predicted by the model.
MRR~\cite{Craswell2009} is the average of the reciprocal ranks over a sample of passes $P$:
\begin{equation}
  \textrm{MRR} = \frac{1}{|P|}\displaystyle\sum_{p=1}^{|P|}\frac{1}{\textrm{rank}_p}
\end{equation}
where $\textrm{rank}_p$ is the rank of the correct receiver for the $p$th pass.
The reciprocal value of MRR corresponds to the harmonic mean of the ranks.
MRR ranges from 0 to 1, the larger the better. 
%A larger MRR means the correct receiver is closer to the front of the predicted ranked list.
While top-N accuracy captures how likely the correct receiver appears in the top-N predicted receivers, 
MRR captures the average rank of the correct receiver in the predicted list of receiver candidates.
 
%We use \textbf{10-fold cross-validation} to estimate the efficacy of our models. All the passes in the dataset are randomly partitioned into 10 sets of roughly equal size. One subset is used as testing set (i.e., the held-out set) and the other nine subsets are used as training set. 
%We train our models using the training set and evaluate the performance of our models on the held-out set.
%The process repeats 10 times until all subsets are used as testing set once.

\subsection{Feature importance}

In order to understand the importance of the features in our model, we use the feature importance scores that are automatically provided by a trained LightGBM model.
GBDT (e.g., LightGBM) provides a straightforward way to retrieve the importance scores of each feature~\cite{friedman2001elements}.

After the boosting decision trees are constructed, for each decision tree, the importance of a feature is calculated by the amount that the feature improves the performance measure (i.e., Gini index) at its split point.
The importance of each feature is then averaged across all of the decisions trees in the model. 



\section{Results} \label{results}

This section discusses the answers to our research questions.
\vspace{-0.2cm}

\subsection{RQ1: How well can we model the receiver of a pass?}\label{RQ2-results}

\textbf{Our model can predict the receiver of a pass with a top-1, top-3 and top-5 accuracy of 50\%, 84\%, and 94\%, respectively, when we exclude false passes (i.e., passes to the other team).}
Table~\ref{tab:performance-accurate-passes} shows the performance of our model when we exclude false passes. 
The ``Back-field'', ``Middle-field'', ``Front-field'' and ``Overall'' columns show the performance of our model for back-field passes, middle-field passes, front-field passes and all passes, respectively. 
A top-3 accuracy of 84\% for all passes means that the actual receiver of a pass has a 84\% chance to appear in our top-3 predicted candidates.
The MRR value for all passes is 0.68, which means on average, the correct receiver is ranked $1.5$th (i.e., 1/0.68) out of 10 or less receiver candidates (i.e., all teammates of the sender).

%\setlength{\tabcolsep}{3pt}
\begin{table}[!t]
\caption{The accuracy of our model for predicting the receiver of a pass (excluding false passes).}
\centering
\renewcommand{\tabcolsep}{3pt}
\begin{tabular}{lcccc}
  \toprule
  & Back-field & Middle-field & Front-field & Overall \\
  \midrule
  Top-1 accuracy & 53\% & 46\% & 55\% & 50\% \\
  Top-3 accuracy & 84\% & 81\% & 91\% & 84\% \\
  Top-5 accuracy & 93\% & 93\% & 97\% & 94\% \\
  MRR & 0.70 & 0.66 & 0.73 & 0.68 \\
  \bottomrule
\end{tabular}
\label{tab:performance-accurate-passes}
%\vspace{-0.4cm}
\end{table}

\begin{table}[!t]
\caption{Comparing the accuracy of our model with baseline models (excluding false passes).}
\centering
\renewcommand{\tabcolsep}{3pt}
\begin{threeparttable}
\begin{tabular}{lcccc}
  \toprule
  & LightGBM & Random Guess\tnote{1} & NearestPass\tnote{2} & PassForwards\tnote{3} \\
  \midrule
  Top-1 accuracy & \textbf{50\%} & 10\% & 33\% & 27\% \\
  Top-3 accuracy & \textbf{84\%} & 30\% & 70\% & 54\% \\
  Top-5 accuracy & \textbf{94\%} & 50\% & 86\% & 71\% \\
  MRR & \textbf{0.68} & 0.29 & 0.55 & 0.47 \\
  \bottomrule
\end{tabular}
\begin{tablenotes}
\item[1] Randomly guess a candidate receiver.
\item[2] A model that always passes the ball to the nearest teammate of the sender.
\item[3] A model that always passes the ball to the nearest teammate that is in the forward direction relative to the sender.
\end{tablenotes}
\end{threeparttable}
\label{tab:comparing-performance-accurate-passes}
%\vspace{-0.4cm}
\end{table}

\textbf{Our model can predict the receiver of a pass with a top-1, top-3 and top-5 accuracy of 41\%, 70\%, and 81\%, respectively, when we consider all passes.}
Table~\ref{tab:performance-all-passes} shows the performance of our model when we consider all passes (including false passes). 
The performance of our model decreases when we consider false passes (i.e., passes to the other team). 
False passes are very difficult to predict because it is not the sender player's intention to pass the ball to the other team. 
The MRR value for all passes is 0.58, which means the correct receiver is averagely ranked $1.7$th (i.e., 1/0.58) out of all 21 or less candidate receivers (i.e., all players excluding the sender).

%\setlength{\tabcolsep}{6pt}
\begin{table}[!t]
%\vspace{-0.2cm}
\caption{The accuracy of our model for predicting the receiver of a pass (considering all passes including passes to the other team).}
\centering
\renewcommand{\tabcolsep}{3pt}
\begin{tabular}{lcccc}
  \toprule
  & Back-field & Middle-field & Front-field & Overall \\
  \midrule
  Top-1 accuracy & 45\% & 38\% & 43\% & 41\% \\
  Top-3 accuracy & 72\% & 68\% & 72\% & 70\% \\
  Top-5 accuracy & 82\% & 80\% & 83\% & 81\% \\
  MRR & 0.61 & 0.56 & 0.60 & 0.58 \\
  \bottomrule
\end{tabular}
\label{tab:performance-all-passes}
%\vspace{-0.4cm}
\end{table}

\begin{table}[!t]
\caption{Comparing the accuracy of our model with baseline models (considering all passes including passes to the other team).}
\centering
\renewcommand{\tabcolsep}{3pt}
%\begin{threeparttable}
\begin{tabular}{lcccc}
  \toprule
  & LightGBM & Random Guess & NearestPass & PassForwards \\
  \midrule
  Top-1 accuracy & \textbf{41\%} & 5\% & 27\% & 23\% \\
  Top-3 accuracy & \textbf{70\%} & 14\% & 58\% & 45\% \\
  Top-5 accuracy & \textbf{81\%} & 24\% & 71\% & 59\% \\
  MRR & \textbf{0.58} & 0.17 & 0.47 & 0.40 \\
  \bottomrule
\end{tabular}
%\begin{tablenotes}
%\item[1] Randomly guess a candidate receiver.
%\item[2] A model that always passes the ball to the nearest teammate of the sender.
%\item[3] A model that always passes the ball to the nearest teammate that is in the forward direction relative to the sender.
%\end{tablenotes}
%\end{threeparttable}
\label{tab:comparing-performance-all-passes}
%\vspace{-0.4cm}
\end{table}

\textbf{Our model performs better for back-field and front-field passes, while performing worse for middle-field passes.}
Table~\ref{tab:performance-accurate-passes} and Table~\ref{tab:performance-all-passes} also shows the performance of our model for back-field, middle-field and front-field passes, separately.
Surprisingly, the performance of our model is the worst for middle-field passes. A player in the middle area may have more passing options, thereby increasing the difficulty to predict the right receivers.

%\vspace{-0.4cm}
\hypobox{
Our model can predict the receiver of a pass with a top-1, top-3 and top-5 accuracy of 50\%, 84\%, and 94\%, respectively, when we exclude false passes. Our model performs better when the sender of a pass is in the back or front area of the field.
}

\subsection{RQ2: how well can we model passing targets?}\label{RQ2-results}


%\subsection{RQ3: what are the influential factors that explain the passing targets?}\label{RQ3-results}



\section{Conclusions} \label{conclusions}

This work proposes a novel approach to predict the receivers of football passes.
We analyze a dataset containing 12,124 passes from 14 real-world football games and discuss players' passing characteristics. We find that players present different passing characteristics in different areas of the field.
We then extract 54 features along five dimensions and build a LightGBM model to predict the receiver of a pass. 
Our model achieves a top-1, top-3, and top-5 accuracy of 50\%, 84\%, and 94\%, respectively, when we exclude false passes.
Our model outperforms three baseline models that we use to rank the candidate receivers of a pass.
We find that the features that capture the positions of the candidate receivers play the most important roles in explaining the receiver of a pass.
%We believe that our approaches and findings can help football practitioners understand the characteristics of football passes and make informed tactical decisions.
We believe that our approaches and findings can help football practitioners better understand the factors that impact the receiver of a pass and make informed tactical decisions.

%
% ---- Bibliography ----
%
% BibTeX users should specify bibliography style 'splncs04'.
% References will then be sorted and formatted in the correct style.
%
\bibliographystyle{splncs04}
\bibliography{mybibliography}
%

\end{document}
