% This is samplepaper.tex, a sample chapter demonstrating the
% LLNCS macro package for Springer Computer Science proceedings;
% Version 2.20 of 2017/10/04
%
\documentclass[runningheads]{llncs}
%
\usepackage{graphicx}
% Used for displaying a sample figure. If possible, figure files should
% be included in EPS format.
%
% If you use the hyperref package, please uncomment the following line
% to display URLs in blue roman font according to Springer's eBook style:
% \renewcommand\UrlFont{\color{blue}\rmfamily}

\usepackage{color}
\newcommand{\heng}[1]{\textcolor{blue}{{\it [Heng says: #1]}}}
\newcommand{\todo}[1]{\textcolor{red}{\textbf{[[TODO: #1]]}}}

\usepackage{fancybox}		
\newcommand{\hypobox}[1]{\begin{center}%	
	\noindent\thicklines\setlength{\fboxsep}{7pt}%	
	%\cornersize{0}\Ovalbox{\begin{minipage}{3.1in}%
	%\cornersize{0}\shadowbox{\begin{minipage}{3.1in}%	
	%\cornersize{0}\Ovalbox{\begin{minipage}{4in}%
	\cornersize{0}\Ovalbox{\begin{minipage}{0.96\textwidth}%	
		%\vspace{-0.1cm}
		%\textit{#1}
		#1
		%\vspace{-0.1cm}
		\end{minipage}} \end{center}}

%\usepackage{booktabs} % For formal tables
%\usepackage{url}

\begin{document}
%
\title{Predicting the Receivers of Football Passes}
%
%\titlerunning{Abbreviated paper title}
% If the paper title is too long for the running head, you can set
% an abbreviated paper title here
%
\author{Heng Li\inst{1} \and
Zhiying Zhang\inst{2}
% \and Third Author\inst{3}\orcidID{2222--3333-4444-5555}
 }
%
\authorrunning{H. Li and Z. Zhang}
% First names are abbreviated in the running head.
% If there are more than two authors, 'et al.' is used.
%
\institute{School of Computing, Queen's University, Kingston, Canada\\
\email{hengli@cs.queensu.ca}
\and
Microsoft, Bellevue, WA, USA\\
\email{zhiyingz@microsoft.com}\\
}
%
\maketitle              % typeset the header of the contribution
%
\begin{abstract}
%The abstract should briefly summarize the contents of the paper in
%150--250 words.
Football is a highly-collaborative sports. 
Passing the ball to the right person is essential for winning a football game.
Anticipating the target of a pass can help football players build better collaborations on the field and help coaches make better tactic decisions.
In this work, we use machine learning models to learn how professional football players pass the ball, and leverage the models to predict the targets of football passes.
Our models can predict the passing target with a top-1 accuracy of 41\% and a top-5 accuracy of 82\% (\todo{numbers to be updated}).
This paper discusses the features that we use to build our models, our modeling approaches, and the important factors that explain the the target of a pass.


\keywords{Football pass prediction \and Learning to rank \and LambdaMART \and Gradient boosting decision tree \and LightGBM.}
\end{abstract}
%
%
%

\section{Introduction} \label{intro}
%This section discuss our motivation, briefly mention our approach, and introduce the research questions that we want to answer:
In a football game, players pass the ball to their teammates in order to create good shooting opportunities or prevent the opposing team from getting the control of the ball.
Accurately passing ball to the right player is essential for winning a football game~\cite{Ali2011Measuring,Hughes2005Analysis}.

Prior work~\cite{reep1968skill,Hughes2005Analysis} studies how passing sequences lead to goals. Their findings have shaped the tactics of many football coaches.
In this work, we study football players' passing patterns and construct machine learning models to anticipate passing targets.
We believe that football coaches and players can take our results into consideration when they make their tactics or make their passes/runs. For example, anticipating passing targets can help players be aware of the best positions to get a pass.

This work analyzes a dataset which contains 12,124 passes performed by a Belgian football club in 14 games. We want to answer the following research questions: 

\begin{description}
	\item \textbf{RQ1: what are players' passing patterns?}
	
	For example, most of the passes are passing forwards.
	
	\item \textbf{RQ2: how well can we model passing targets?}
	\item \textbf{RQ3: what are the influential factors that explain the passing targets?}
\end{description}

\section{Data Exploration and Pre-processing}

\textbf{Overall, players' passing accuracy is 83\%, and the passing accuracy decreases from the back field to the front field.} 
Table~\ref{tab:pass-statistics} shows a summary of players' passing statistics. 
We define the passing accuracy as the ratio of the passes that reach a teammate.
We divide the field into three equal-sized areas along the long side of the field, namely back field, middle field and front field.
We define a pass as a \textbf{back-field pass}, \textbf{middle-field pass}, or \textbf{front-field} pass when the sender is within the back field, the middle field and the front field, respectively.
The passing accuracy for the back field, middle field, and front field is 86\%, 83\%, and 79\%, respectively.

\begin{table}[!t]
%\vspace{-0.2cm}
\caption{A summary of players' passing statistics.}
\centering
\begin{tabular}{lcccc}
  \hline
  & Back-field & Middle-field & Front-field & Overall \\
  \hline
  Passing accuracy & 86\% & 83\% & 79\% & 83\% \\
  Median passing distance (m) & 17 & 14 & 11 & 14 \\
  Passing forwards ratio & 74\% & 61\% & 50\% & 62\% \\
  \hline
\end{tabular}
\label{tab:pass-statistics}
%\vspace{-0.2cm}
\end{table}

\textbf{The median passing distance is 14 meters, and the passing distance decreases from the back field to the front field.}
Table~\ref{tab:pass-dist} shows the five-number summary of players' passing distance. While the maximum passing distance is 70 meters, 50\% of the passes are between 9 and 20 meters.
As shown in Table~\ref{tab:pass-statistics}, the median passing distance for the back field, middle field, and front field is 17, 14, and 11 meters, respectively.

\begin{table}[!t]
%\vspace{-0.1cm}
\caption{Five-number summary of players' passing distance.}
\centering
\begin{tabular}{c c c c c}
\hline
Min. & 1st Qu. & Median & 3rd Qu. & Max. \\
\hline
0 & 9 & 14 & 20 & 70 \\
\hline
\end{tabular}
\label{tab:pass-dist}
%\vspace{-0.2cm}
\end{table}

\textbf{Players pass the ball forwards in 62\% of the passes, and the ratio of forward-passing decreases from the back field to the front field.} In the back field, players pass the ball forwards in 74\% of the passes, and the ratio decreases to 61\% and 50\% for middle-field passes and front-field passes, respectively.

%\vspace{-0.4cm}
\hypobox{
Players present different passing characteristics in different areas of the field. 
%which motivates us to extract features that capture the positions of the players in the field. 
Such differences suggest us to build and evaluate our model in different areas of the field separately.
}

\section{Methodology}
This section discuss our overall methodology, our feature extraction, feature selection, modeling approaches, and evaluation approaches.

\section{Results} \label{results}

This section discuss our answers to our research questions.

\subsection{RQ1: what are players' passing patterns?}\label{RQ1-results}
	
\subsection{RQ2: how well can we model passing targets?}\label{RQ2-results}

\subsection{RQ3: what are the influential factors that explain the passing targets?}\label{RQ3-results}


\section{Conclusions} \label{conclusions}

This work proposes a novel approach to predict the receivers of football passes.
We analyze a dataset containing 12,124 passes from 14 real-world football games, and discusses players' passing patterns. We find that players present different passing patterns in different areas of the field.
We then extract 54 features along five dimensions and build a LightGBM model to predict the target of a pass. 
Our model achieves a top-1, top-3, and top-5 accuracy of 50\%, 84\%, and 94\%, respectively, when we exclude false passes.
We believe that our approaches and findings can help football practitioners understand the patterns of football passes and make informed tactics in football games.

%
% ---- Bibliography ----
%
% BibTeX users should specify bibliography style 'splncs04'.
% References will then be sorted and formatted in the correct style.
%
\bibliographystyle{splncs04}
\bibliography{mybibliography}
%

\end{document}
