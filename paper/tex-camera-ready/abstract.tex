Football (or association football) is a highly-collaborative team sport. 
Passing the ball to the right player is essential for winning a football game.
Anticipating the receiver of a pass can help football players build better collaborations and help coaches make informed tactical decisions.
In this work, we analyze a public dataset that contains 12,124 passes performed by professional football players.
%First, we study the characteristics of the passes, e.g., the distance of the passes.
We extract five dimensions of features from the dataset and build a learning to rank model to predict the receiver of a pass. 
Our model's first, top-3 and top-5 guesses find the correct receiver of a pass with an accuracy of 50\%, 84\%, and 94\%, respectively, when we exclude false passes, 
which outperforms three baseline models that we use to rank the candidate receivers of a pass.
%We compare our model with three baseline models and we find that our model outperforms the baseline models. 
%Our model outperforms three baseline models that we use to rank the candidate receivers of a pass.
The features that capture the positions of the candidate receivers play the most important roles in explaining the receiver of a pass.

