\section{Data Exploration and Pre-processing}

\textbf{Overall, players' passing accuracy is 83\%, and the passing accuracy decreases from the back field to the front field.} 
Table~\ref{tab:pass-statistics} shows a summary of players' passing statistics. 
We define the passing accuracy as the ratio of the passes that reach a teammate.
We divide the field into three equal-sized areas along the long side of the field, namely back field, middle field and front field.
We define a pass as a \textbf{back-field pass}, \textbf{middle-field pass}, or \textbf{front-field} pass when the sender is within the back field, the middle field and the front field, respectively.
The passing accuracy for the back field, middle field, and front field is 86\%, 83\%, and 79\%, respectively.

\begin{table}[!t]
%\vspace{-0.2cm}
\caption{A summary of players' passing statistics.}
\centering
\begin{tabular}{lcccc}
  \hline
  & Back-field & Middle-field & Front-field & Overall \\
  \hline
  Passing accuracy & 86\% & 83\% & 79\% & 83\% \\
  Median passing distance (m) & 17 & 14 & 11 & 14 \\
  Passing forwards ratio & 74\% & 61\% & 50\% & 62\% \\
  \hline
\end{tabular}
\label{tab:pass-statistics}
%\vspace{-0.2cm}
\end{table}

\textbf{The median passing distance is 14 meters, and the passing distance decreases from the back field to the front field.}
Table~\ref{tab:pass-dist} shows the five-number summary of players' passing distance. While the maximum passing distance is 70 meters, 50\% of the passes are between 9 and 20 meters.
As shown in Table~\ref{tab:pass-statistics}, the median passing distance for the back field, middle field, and front field is 17, 14, and 11 meters, respectively.

\begin{table}[!t]
%\vspace{-0.1cm}
\caption{Five-number summary of players' passing distance.}
\centering
\begin{tabular}{c c c c c}
\hline
Min. & 1st Qu. & Median & 3rd Qu. & Max. \\
\hline
0 & 9 & 14 & 20 & 70 \\
\hline
\end{tabular}
\label{tab:pass-dist}
%\vspace{-0.2cm}
\end{table}

\textbf{Players pass the ball forwards in 62\% of the passes, and the ratio of forward-passing decreases from the back field to the front field.} In the back field, players pass the ball forwards in 74\% of the passes, and the ratio decreases to 61\% and 50\% for middle-field passes and front-field passes, respectively.

We pre-process the data as follows. For each line in the original pass data, we drop this line if the sender's coordinates are missing, but we don't drop it if only the receiver's coordinates are missing, because we can still learn from this pass which players are not good candidates to pass to. For each pass, we compute features between the sender and each candidate if the candidate's coordinate is present.

%\vspace{-0.4cm}
\hypobox{
Players present different passing characteristics in different areas of the field. 
%which motivates us to extract features that capture the positions of the players in the field. 
Such differences suggest us to build and evaluate our model in different areas of the field separately.
}